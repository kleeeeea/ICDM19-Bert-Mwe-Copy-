%!TEX root =  concept_mining.tex


\section{Related work}
\subsection{text modeling for neural network}
has become prevalent in recent years, due to their ability in capturing the meaning of words, as shown in 
tasks such as word similarity, analogy, relation prediction \cite{turian2010word, pennington2014glove, mikolov2013distributed}.
Furthermore, it has become the dominant, if not the only approach, for deep learning models to process raw text into numerical form, and feed into downstream neural network modules to solve various kinds of NLP tasks  \cite{collobert2011natural, wu2016google, conneau2017very, severyn2015learning}.
A lot of methods has been proposed to modify the original word embedding. One recent trend is to go "small", and try to exploit sub-word level information, such as character, or sub-sequence of characters inside a word \cite{wieting2016charagram, bojanowski2016enriching, pinter2017mimicking}, by borrowing information from words with similar subsequence of characters, they are able to infer the meaning of rare words and use it for downstream tasks. 
Others have kept the unigram formulation, and try to better learn the embedding by extending it to multiple word sense \cite{neelakantan2015efficient, athiwaratkun2017multimodal}, 
different type of contexts \cite{levy2014dependency, melamud2016role}, 
or try to incorporate more information, such as knowledge base hierarchical and similarity \cite{nickel2017poincare, faruqui2014retrofitting}, 
or to pass it through a language modeling layers before feeding to downstream neural network in order to incorporate context information \cite{mccann2017learned, peters2018deep}.
Finally, there is also a trend of going "big", to embed multi-word phrases, or even sentence or paragraphs into vector space \cite{mikolov2013distributed, hill2015learning, arora2016simple} . 
However, these works mostly focus on \emph{intrinsic evaluation} of the embedding itself, using tasks such as similarity, but not on applying it into deep learning models.

\subsection{entity and concept discovery in text}
The existence of atomic, non-composable phrases has been widely recognized in the NLP community with continued research interest \cite{sag2002multiword, baldwin2010multiword, constant2017multiword, hashimoto2016adaptive}. \cite{jackendoff1997architecture}
the number of MWEs in a speaker's lexicon is of the same order of magnitude as the number of single words,
%claims that our mental lexicon (vocabulary) of MWEs is as extensive as the one that is made up of single words, 
Specialized domain vocabulary, such as terminology, overwhelmingly consists of MWEs
and \cite{justeson1995technical} claims that 
"when native English forms are used to create new terms, it most often takes at least two words to adequately specify a meaning", and that "often, well established one-word terms are Greek or Latin forms made up of more than one root, e.g. aerodynamics".
In fact, researches in information extraction and text mining has been directly using the occurrences of these multi word structure, e.g. based on the POS-tag patterns, to harvest natural concepts or entities from text data, and to further construct structured knowledge bases \cite{banko2007open, wu2012probase, ren2016automatic}. 
Moreover, phrases, or n-gram, has been widely used as features for traditional NLP models that heavily rely on feature engineering \cite{furnkranz1998study} or for certain application such as statistical machine translation or parsing \cite{koehn2009statistical, cafferkey2007multi, savary2015parseme}. There are also phrase based neural network model that has been designed, especially in the area of machine translation \cite{cho2014learning, huang2017neural}. However their model is specific to the application and in general not applicable to other NLP tasks.

\subsection{downstream application of text non-compositionality}
Another line of related work is on 
concept/entity aware representation
%\subsection{Information Retrieval}
, which utilize information from a knowledge base, such as entity description, entity type or links to other entities to perform query expansion \cite{xiong2015query, dalton2014entity}, 
and improves relevance computation \cite{shen2018entity}, or ranking \cite{xiong2015esdrank} .
Explicitly annotated entities in documents have been utilized to represent text \cite{xiong2017explicit, xiong2017word}, which are then used to derive features to train the ranking model. However, directly adapting these information retrieval approaches to hierarchical classification setting, where one needs to assign each document into the correct category, is rather challenging as the query needs to cover the content of the entire scientific field.

Entities, such as people, locations, or abstract concepts, are natural units for organizing and retrieving information [10]. Previous
studies found that over 70% of Bing’s query and more than 50%
of traffic in Semantic Scholar are related to entities [12, 39]. The
recent availability of large-scale knowledge repositories and accurate entity linking tools have further motivated a growing body of
work on entity-aware ranking models. These models can be roughly
categorized into three classes: expansion-based, projection-based,
and representation-based.
The expansion-based methods use entity descriptions from knowledge repositories to enhance query representation. Xu et al. [40]
use entity descriptions in Wikipedia as pseudo relevance feedback
corpus to obtain cleaner expansion terms; Xiong and Callen [36]
utilize the description of Freebase entities related to the query for
query expansion; Dalton et al. [7] expand a query using the text
fields of the attributes of the query-related entities and generate
richer learning-to-rank features based on the expanded texts.
The projection-based methods try to project both query and document onto an entity space for comparison. Liu and Fang [20] use
entities from a query and its related documents to construct a latent
entity space and then connect the query and documents based on
the descriptions of the latent entities. 
